\documentclass{beamer}
\usetheme{default}
\usepackage{import}
\usepackage{xpatch}
\makeatletter
\patchcmd\beamer@@tmpl@frametitle{\insertframetitle}{\insertsection-\insertframetitle}{}{}
\makeatother
\setbeamertemplate{frametitle continuation}[from second] 
\title{Shell e GNU básico}
\author{João Pedro Abreu de Souza}
\begin{document}
\maketitle
\section{Preparação}
\begin{frame}{Perguntas Básicas}
	\begin{itemize}
		\item Quantos aqui já utilizaram GNU
		\pause
		\item Quantos aqui já programaram em C
		\pause
		\item Quantos aqui já fizeram Sistemas Operacionais
	\end{itemize}
\end{frame}
\begin{frame}{Organização do minicurso}
 \begin{itemize}
     \item Conceitos básicos
     \item Tarefas
 \end{itemize}
\end{frame}
\section{Conceitos básicos}
\begin{frame}{História Unix}
	\begin{itemize}
		\item Ken Thompson e Dennis Ritchie em 1969
		\item Assembly para pdp-7
		\item portar pdp-11
	\end{itemize}
\end{frame}
\begin{frame}{História GNU}
	\begin{itemize}
		\item Richard Matthew Stallman no MIT em 1984
		\item projetou 
		\item portar pdp-11
	\end{itemize}
\end{frame}
\begin{frame}{filosofia Unix}
	\begin{itemize}
		\item Simplicidade: Desenvolver programas que façam uma coisa e façam bem. Escrever programas que interajam.
		\item Composição: Escrever programas para serem conectados com outros programas. Projetar programas que manipulem texto, pois este é um formato universal.
		\item Modularidade: Projetar programas para serem construídos em torno de um núcleo forte. Escrever programas que possam ser estendidos, em vez de serem alterados.
	\end{itemize}
\end{frame}
\section{Tarefas}
\begin{frame}{tarefas}
	
\end{frame}
\end{document}
